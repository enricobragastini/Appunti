\documentclass[11pt,oneside,draft]{book}

% PACCHETTI
\usepackage{tabto}              % strumento per inserire tab nel testo
\usepackage[                    % geometria della pagina
    a4paper,
    inner=1.8cm,
    outer=1.8cm,
    top=3cm,
    bottom=3cm,
    bindingoffset=1.2cm
]{geometry}
\usepackage[utf8]{inputenc}     % 3 pacchetti per l'italiano
\usepackage[italian]{babel}
\usepackage[T1]{fontenc}
\usepackage{titlesec}           % custom chapter titles

\usepackage{fancyhdr}

% GRAFICI
\usepackage{tikz}
\usepackage{amsmath}
\tikzset{
  pics/linegraph/.style = {
    code = {
       \def\mymax{-1000}
       \def\mymin{1000}
       \foreach \y in {#1} { % determine min and max values
         \pgfmathparse{max(\y,\mymax)}\xdef\mymax{\pgfmathresult}
         \pgfmathparse{min(\y,\mymin)}\xdef\mymin{\pgfmathresult}
       }
       \foreach \y [count=\c] in {#1} {
           % use \pgfmathparse to determine the appropriate label
           \pgfmathparse{\y==\mymax}
           \ifnum\pgfmathresult=1\def\mylabel{\text{max}}
           \else
             \pgfmathparse{\y==\mymin}
             \ifnum\pgfmathresult=1\def\mylabel{\text{min}}
             \else\def\mylabel{\c}
             \fi
           \fi
           % draw the line
           \draw [->,densely dashed](\c,0)--++(90:\y) node[above]{$x_{\mylabel}$};
       }
    }
  }
}


% INFORMAZIONI SUL DOCUMENTO
\title{\Large{\textbf{Probabilità e Statistica}}}
\author{Enrico Bragastini}
\titleformat{\chapter}[display]{\normalfont\bfseries}{}{0pt}{\LARGE}


% CONTENUTO
\begin{document}
\pagestyle{fancy}
\fancyhf{}
\rhead{}
\lhead{\nouppercase\leftmark}
\cfoot{\thepage}
\frontmatter

% Prima pagina - Titolo
\maketitle
\tableofcontents

\mainmatter

% STATISTICA DESCRITTIVA
\chapter{Statistica Descrittiva}
La \textbf{statistica descrittiva} è la branca della statistica che studia i criteri di rilevazione,
classificazione, sintesi e rappresentazione dei dati appresi dallo studio di una popolazione o di
una parte di essa (detta \textit{campione}).

I risultati ottenuti nell'ambito della statistica descrittiva si possono definire \emph{certi}, a meno di errori
di misurazione dovuti al caso, che sono in media pari a zero. Da questo punto di vista si differenzia dalla \emph{statistica inferenziale}, alla quale sono associati inoltre errori di valutazione.

\section{Descrizione degli insiemi di dati}
\subsection{Frequenza statistica}
In statistica esistono 2 tipologie di frequenze:
\begin{enumerate}
    \item \textbf{Frequenza Assoluta}: è il numero di volte che si verifica un evento a prescindere dal numero totale delle prove.
    \item \textbf{Frequenza Relativa}: è il rapporto tra la frequenza assoluta e il numero di prove eseguite; viene misurata con un numero decimale compreso tra 0 e 1, o in percentuale.
          Considerando un insieme di \emph{n} dati, se \emph{f} rappresenta la frequenza assoluta di un dato, $f/n$ indica la sua \emph{Frequenza Relativa}.
\end{enumerate}

\subsection{Tabella di Frequenza}
Un set di dati relativamente piccolo può essere rappresentato in una \textbf{tabella di frequenza}.

~\newline
\underline{Esempio:} \newline
\noindent Vengono raccolti nella \emph{tabella di frequenza} i valori degli stipendi annuali di
ingegneri neolaureati

\begin{center}
    \bgroup
    \def\arraystretch{1.5}
    \begin{tabular}{ |c| c c c c c c c c c c c|}
        \hline
        SALARIO ANNUALE ($*10^3$) & 47 & 48 & 49 & 50 & 51 & 52 & 53 & 54 & 56 & 57 & 60 \\
        \hline
        FREQUENZA                 & 4  & 1  & 3  & 5  & 8  & 10 & 0  & 5  & 2  & 3  & 1  \\
        \hline
    \end{tabular}
    \egroup
\end{center}

\subsection{Classi di Appartenenza (bins)}
Quando il numero di valori distinti è troppo grande, si suddividono questi valori in un certo
numero di \textbf{classi di appartenenza}, o di \emph{intervalli}.

È indispensabile trovare un corretto \textbf{equilibrio}, scegliendo la quantità corretta di \emph{classi di appartenenza}:

- Scegliere \emph{poche classi} comporta una \emph{alta perdita di informazioni}.

- Scegliere \emph{troppe classi} comporta la costruzione di un \emph{istogramma frastagliato e poco informativo}.


\subsection{Grafici}
\subsubsection{Line Graph (grafico a bastoncini)}
Permette di visualizzare i dati con degli \emph{stem}

\begin{tikzpicture}
    % Draw the axes
    \draw [<->] (0,4)node[right]{$y$}|-(8,0)node[right]{$x$};

    % draw the line graph
    \pic{linegraph={}};
\end{tikzpicture}










\end{document}