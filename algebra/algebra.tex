\documentclass[12pt,oneside]{book}

% PACCHETTI
\usepackage{tabto}              % strumento per inserire tab nel testo
\usepackage[                    % geometria della pagina
    a4paper,
    inner=2cm,
    outer=3cm,
    top=3cm,
    bottom=3cm,
    bindingoffset=1.2cm
]{geometry}
\usepackage[utf8]{inputenc}     % 3 pacchetti per l'italiano
\usepackage[italian]{babel}
\usepackage[T1]{fontenc}
\usepackage{titlesec}           % custom chapter titles

\usepackage{fancyhdr}
\usepackage{amsmath}
\usepackage{xcolor}


% INFORMAZIONI SUL DOCUMENTO
\title{\Large{\textbf{Algebra Lineare}}}
\author{Enrico Bragastini}
\titleformat{\chapter}[display]{\normalfont\bfseries}{}{0pt}{\LARGE}


% CONTENUTO
\begin{document}
\pagestyle{fancy}
\fancyhf{}
\rhead{}
\lhead{\nouppercase\leftmark}
\cfoot{\thepage}
\frontmatter

% Prima pagina - Titolo
\maketitle
\tableofcontents

\mainmatter
\chapter{Matrici e sistemi lineari}

\section{Matrici}
Una \textbf{Matrice} è una tabella numerica a doppia entrata con i coefficienti ordinati per righe
e per colonne

\subsection{Dimensioni di una matrice (forma)}
Si dice che una matrice è $m \times n$ se ha \emph{m righe} e \emph{n colonne}.
~\newline
Per esempio, date le seguenti due matrici:
\begin{equation*}
    A =
    \begin{bmatrix}
        1 & 2 & 3 \\
        4 & 5 & 6 \\
    \end{bmatrix}
    \; B =
    \begin{bmatrix}
        1 & 2 & 3 \\
        4 & 5 & 6 \\
    \end{bmatrix}
\end{equation*}
La matrice A è $2 \times 3$ perchè ha 2 righe e 3 colonne. \newline La matrice B è $2 \times 2$ perché ha 2 righe
e 2 colonne

\subsection{Vettore riga e Vettore colonna}
Esistono due particolari tipologie di matrici distinte dalla loro \emph{forma}:
\begin{itemize}
    \item \textbf{Vettore Riga} \newline
          Si tratta di una matrice composta da \emph{una sola riga}. Un vettore riga è quindi una matrice
          di forma $1 \times n$.
          \begin{equation*}
              A =
              \begin{bmatrix}
                  1 & 2 & 3 \\
              \end{bmatrix}
          \end{equation*}
    \item \textbf{Vettore Colonna} \newline
          Si tratta di una matrice composta da \emph{una sola colonna}. Un vettore colonna è quindi una matrice
          di forma $m \times 1$.
          \begin{equation*}
              A =
              \begin{bmatrix}
                  1 \\ 2 \\ 3 \\
              \end{bmatrix}
          \end{equation*}
\end{itemize}

\newpage
\subsection{Matrice quadrata}
Una Matrice si dice \textbf{quadrata} quando il numero delle righe è uguale al numero delle colonne, ovvero quando
$m = n$. In tale caso, $n$ è chiamato \textbf{ordine} della matrice.

~\newline
Esempio
\begin{equation*}
    A =
    \begin{bmatrix}
        \color{red} 1  & \color{blue}2 \\
        \color{blue} 3 & \color{red} 4 \\
    \end{bmatrix}
\end{equation*}
La matrice A è \emph{quadrata} di \emph{ordine} pari a 2. Gli elementi della matrice in \textcolor{red}{rosso}
fanno parte della \textbf{diagonale principale}. Gli elementi della matrice in \textcolor{blue}{blu} fanno
parte della \textbf{diagonale secondaria}

\subsection{Posto in una matrice}
Ogni elemento di una matrice è univocamente determinato dal posto che occupa nella tabella.
L'unico elemento di posto $(i,j)$ è l'elemento che si trova nella \emph{i-esima riga} e nella
\emph{j-esima colonna}. \newline
~\newline
Esempio:
\begin{equation*}
    A =
    \begin{bmatrix}
        1 & 2 & 3 \\
        4 & 5 & 6 \\
    \end{bmatrix}
\end{equation*}
Nella matrice A:
\begin{itemize}
    \item 1 è l'elemento di posto (1, 1)
    \item 2 è l'elemento di posto (1, 2)
    \item 6 è l'elemento di posto (2, 3)
\end{itemize}

\subsection{Notazione generica}
Una matrice A di forma $m \times n$, ovvero una matrice del tipo:
\begin{equation*}
    A =
    \begin{bmatrix}
        a_{1,1} & a_{1,2} & \cdots & a_{1,n} \\
        a_{2,1} & a_{2,2} & \cdots & a_{2,n} \\
        \vdots  & \vdots  & \ddots & \vdots  \\
        a_{m,1} & a_{m,2} & \cdots & a_{m,n}
    \end{bmatrix}
\end{equation*}
può essere indicata mediante la sua \textbf{notazione generica}:
\begin{equation*}
    A = [a_ij] \;
    \begin{align}
        1 \le i \le m & 1 \le j \le n
    \end{align}
\end{equation*}


\end{document}