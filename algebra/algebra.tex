\documentclass[11pt,oneside]{book}

% PACCHETTI
\usepackage{hyperref}           % hyperlinks
\usepackage{tabto}              % strumento per inserire tab nel testo
\usepackage[                    % geometria della pagina
    a4paper,
    inner=2cm,
    outer=3cm,
    top=3cm,
    bottom=3cm,
    bindingoffset=1.2cm,
    headheight=14pt
]{geometry}
\usepackage[utf8]{inputenc}     % 3 pacchetti per l'italiano
\usepackage[italian]{babel}
\usepackage[T1]{fontenc}
\usepackage{titlesec}           % custom chapter titles

\usepackage{fancyhdr}
\usepackage{amsmath, amssymb}
\usepackage{xcolor}
\usepackage{amssymb}


% INFORMAZIONI SUL DOCUMENTO
\title{\Large{\textbf{Algebra Lineare}}}
\author{Enrico Bragastini}
\titleformat{\chapter}[display]{\normalfont\bfseries}{}{0pt}{\LARGE}


% CONTENUTO
\begin{document}
\pagestyle{fancy}
\fancyhf{}
\rhead{}
\lhead{\nouppercase\leftmark}
\cfoot{\thepage}
\frontmatter

% Prima pagina - Titolo
\maketitle
\tableofcontents

\mainmatter
\chapter{Matrici}

\section{Introduzione alle Matrici}
Una \textbf{Matrice} è una tabella numerica a doppia entrata con i coefficienti ordinati per righe
e per colonne

\subsection{Dimensioni di una matrice (forma)}
Si dice che una matrice è $m \times n$ se ha \emph{m righe} e \emph{n colonne}.
~\newline
Per esempio, date le seguenti due matrici:
\begin{equation*}
    A =
    \begin{bmatrix}
        1 & 2 & 3 \\
        4 & 5 & 6 \\
    \end{bmatrix}
    \; \; \; B =
    \begin{bmatrix}
        1 & 2 & 3 \\
        4 & 5 & 6 \\
    \end{bmatrix}
\end{equation*}
- La matrice A è $2 \times 3$ perché ha 2 righe e 3 colonne. \newline
- La matrice B è $2 \times 2$ perché ha 2 righe e 2 colonne

\subsection{Vettore riga e Vettore colonna}
Esistono due particolari tipologie di matrici distinte dalla loro \emph{forma}:
\begin{itemize}
    \item \textbf{Vettore riga} \newline
          Si tratta di una matrice composta da \emph{una sola riga}. Un \emph{vettore riga} è quindi una matrice
          di forma $1 \times n$.
          \begin{equation*}
              A =
              \begin{bmatrix}
                  1 & 2 & 3 \\
              \end{bmatrix}
          \end{equation*}
    \item \textbf{Vettore colonna} \newline
          Si tratta di una matrice composta da \emph{una sola colonna}. Un \emph{vettore colonna} è quindi una matrice
          di forma $m \times 1$.
          \begin{equation*}
              A =
              \begin{bmatrix}
                  1 \\ 2 \\ 3 \\
              \end{bmatrix}
          \end{equation*}
\end{itemize}

\newpage
\subsection{Matrice quadrata}
Una Matrice si dice \textbf{quadrata} quando il numero delle righe è uguale al numero delle colonne, ovvero quando
$m = n$. In tale caso, $n$ è chiamato \textbf{ordine} della matrice.

~\newline
\underline{Esempio:}
\begin{equation*}
    \text{La matrice }A =
    \begin{bmatrix}
        1 & 2 \\
        3 & 4 \\
    \end{bmatrix}
    \text{è \emph{quadrata} di \emph{ordine} pari a 2.}
\end{equation*}

\subsection{Diagonali di una matrice quadrata}
Una \emph{matrice quadrata} presenta due \textbf{diagonali}:
\begin{itemize}
    \item \textbf{Diagonale principale}, costituita dagli elementi che attraversano centralmente la matrice,
          a partire dall'angolo superiore sinistro a quello inferiore destro
    \item \textbf{Diagonale secondaria}, costituita dagli elementi che attraversano centralmente la matrice,
          a partire dall'angolo inferiore sinistro a quello superiore destro
\end{itemize}
~\newline
\underline{Esempio:}
\begin{equation*}
    A =
    \begin{bmatrix}
        \color{red}1  & \color{blue}2 \\
        \color{blue}3 & \color{red}4  \\
    \end{bmatrix}
\end{equation*}
Gli elementi della matrice in \textcolor{red}{rosso}
fanno parte della \textbf{diagonale principale}. Gli elementi della matrice in \textcolor{blue}{blu} fanno
parte della \textbf{diagonale secondaria}

\subsection{Posto in una matrice}
Ogni elemento di una matrice è univocamente determinato dal posto che occupa nella tabella.
L'unico elemento di posto $(i,j)$ è l'elemento che si trova nella \emph{i-esima riga} e nella
\emph{j-esima colonna}. \newline
~\newline
Esempio:
\begin{equation*}
    A =
    \begin{bmatrix}
        1 & 2 & 3 \\
        4 & 5 & 6 \\
    \end{bmatrix}
\end{equation*}
Nella matrice A:
\begin{itemize}
    \item \emph{1} è l'elemento di posto (1, 1)
    \item \emph{2} è l'elemento di posto (1, 2)
    \item \emph{6} è l'elemento di posto (2, 3)
\end{itemize}

\subsection{Notazione generica}
Una matrice A di forma $m \times n$, ovvero una matrice del tipo:
\begin{equation*}
    A =
    \begin{bmatrix}
        a_{1,1} & a_{1,2} & \cdots & a_{1,n} \\
        a_{2,1} & a_{2,2} & \cdots & a_{2,n} \\
        \vdots  & \vdots  & \ddots & \vdots  \\
        a_{m,1} & a_{m,2} & \cdots & a_{m,n}
    \end{bmatrix}
\end{equation*}
può essere indicata mediante la sua \textbf{notazione generica}:

\begin{equation*}
    A = [a_{i,j}] \;\;\;\;\; \begin{aligned} 1&\le i\le m\\ 1&\le j\le n \end{aligned}
\end{equation*}

\subsection{Matrici uguali}
Due matrici si dicono \textbf{uguali} se hanno:
\begin{enumerate}
    \item \textbf{Stessa forma}: \emph{\underline{stesso numero di righe}} e \emph{\underline{stesso numero di colonne}}
    \item \textbf{Stessi coefficienti}
\end{enumerate}
~\newline
\underline{Esempio 1:}
\begin{equation*}
    \begin{bmatrix}
        1 & 2 & 3 \\
        4 & 5 & 6
    \end{bmatrix}
    \neq
    \begin{bmatrix}
        1 & 2 \\
        3 & 4 \\
        5 & 6
    \end{bmatrix}
\end{equation*}
Le due matrici sono diverse perché la prima è $2\times 3$ mentre la seconda è $3\times 2$.

~\newline
\underline{Esempio 2:}
\begin{equation*}
    A=
    \begin{bmatrix}
        1 & 2 \\
        4 & 5
    \end{bmatrix}
    \neq
    \begin{bmatrix}
        1 & 2 \\
        3 & 0 \\
    \end{bmatrix}
    =B
\end{equation*}
Le due matrici sono diverse perché $A_{2,2} \neq B_{2,2}$.

\section{Matrici particolari}
\subsection{Matrice nulla}
La \textbf{matrice nulla} è una matrice $m \times n$ del tipo
\begin{equation*}
    O_{mn} = [0] \;\; \begin{aligned} 1&\le i\le m\\ 1&\le j\le n \end{aligned}
\end{equation*}
tale che
\begin{equation*}
    A + O_{mn} = A
\end{equation*}

\subsection{Matrice opposta}
La \textbf{matrice opposta} di una matrice $A = [a_{i,j}] \;\;$ \tiny$ \begin{aligned} 1&\le i\le m\\ 1&\le j\le n \end{aligned}$ \normalsize  \newline
si denota con $-A$ ed è la matrice con la stessa forma di $A$ (ovvero $m \times n$) tale che
\begin{equation*}
    A + (-A) = O_{mn}
\end{equation*}

\subsection{Matrice identità}
La \textbf{Matrice identità} è una matrice $I$ tale che $AI = A$, dove A è una matrice $n \times m$.
Definiamo quindi la \emph{matrice identità} come la matrice quadrata $n \times n$ composta da tutti 1 sulla diagonale principale
e 0 altrove.
\begin{equation*}
    I_n =
    \begin{bmatrix}
        1      & 0      & \cdots & 0      \\
        0      & 1      & \cdots & 0      \\
        \vdots & \vdots & \ddots & \vdots \\
        0      & 0      & \cdots & 1
    \end{bmatrix}
\end{equation*}

\paragraph{Delta di Kronecker}
In matematica per \emph{delta di Kronecker} si intende una funzione di due variabili discrete
che vale 1 se i loro valori coincidono, mentre vale 0 in caso contrario.
\begin{equation*}
    \delta_{ij} =
    \begin{cases}
        1 \;\;\; i=j \\
        0 \;\;\; i \neq j
    \end{cases}
\end{equation*}
Di conseguenza è possibile definire la \textbf{matrice identità} sfruttando questa funzione:
\begin{equation*}
    I_n = [\delta_{i,j}] \;\;\; \scriptstyle \begin{aligned} 1&\le i\le m\\ 1&\le j\le n \end{aligned}
\end{equation*}

~\newline
\underline{Esempio:}
\begin{equation*}
    \begin{bmatrix}
        1 & 2 & 3 \\
        4 & 5 & 6
    \end{bmatrix}
    \begin{bmatrix}
        1 & 0 & 0 \\
        0 & 1 & 0 \\
        0 & 0 & 1
    \end{bmatrix} = 
    \begin{bmatrix}
        1 & 2 & 3 \\
        4 & 5 & 6
    \end{bmatrix}
\end{equation*}


\section{Operazioni con le matrici}
\subsection{Somma tra matrici}
Se $A$ e $B$ sono due matrici $m \times n$, allora si può definire la loro somma, che viene
denotata con $A+B$.
\begin{equation*}
    \begin{split}
        A & = [a_{i,j}] \;\;\; \begin{aligned} 1&\le i\le m\\ 1&\le j\le n \end{aligned} \\
        \\
        B & = [b_{i,j}] \;\;\; \begin{aligned} 1&\le i\le m\\ 1&\le j\le n \end{aligned}
    \end{split}
    \begin{split}
        A+B & = [c_{i,j}] \;\;\; \begin{aligned} 1&\le i\le m\\ 1&\le j\le n \end{aligned} \;\; \text{dove} \;\;
        c_{i,j} = a_{i,j} + b_{i,j}
    \end{split}
\end{equation*}

La somma di due matrici (con la stessa \emph{forma}) si fa \textbf{posto per posto}.

~\newline
\underline{Esempio:}
\begin{equation*}
    A=
    \begin{bmatrix}
        1 & 2 & 3 \\
        4 & 5 & 6
    \end{bmatrix} \;
    B=
    \begin{bmatrix}
        1 & 0 & 3 \\
        2 & 1 & 0
    \end{bmatrix}
\end{equation*}
\begin{equation*}
    \\ A+B=
    \begin{bmatrix}
        2 & 2 & 6 \\
        6 & 6 & 6
    \end{bmatrix}
\end{equation*}

\subsection{Prodotto per uno scalare}
Sia $\alpha \in \mathbb{R}$ uno scalare.
Consideriamo $A=[a_{i,j}]\;\;\; \begin{aligned} 1&\le i\le m\\ 1&\le j\le n \end{aligned}$
allora $\alpha A$ denota la matrice con la stessa forma di $A$ (ovvero $m \times n$) e con
termine generico $b_{i,j} = \alpha a_{i,j}$

~\newline
\underline{Esempio:}
\begin{align*}
    \alpha   & = 2                                          & A & = \begin{bmatrix}
        1 & 2 \\
        3 & 4
    \end{bmatrix} \;\; 2 \times 2 \\
             &                                              &
    \alpha A & = \begin{bmatrix}
        2 & 4 \\
        6 & 8
    \end{bmatrix} \;\; 2 \times 2
\end{align*}

\paragraph{Tre situazioni particolari} Data $A=[a_{i,j}]$ \tiny$\begin{aligned} 1&\le i\le m\\ 1&\le j\le n \end{aligned}$ \normalsize, allora:
\begin{itemize}
    \item $0 \cdot A = O_{m,n}$ (\emph{matrice nulla})
    \item $1 \cdot A = A$
    \item $(-1) \cdot A = -A$ (\emph{matrice inversa})
\end{itemize}

\subsection{Prodotto tra matrici}
La moltiplicazione tra matrici, a differenza della somma che opera posto per posto, viene effettuata \textbf{riga per colonna}.


\paragraph{Caso base} Prodotto tra un vettore riga e un vettore colonna:
\begin{equation*}
    \begin{bmatrix}
        a_{11} & a_{12} \cdots 1_{1n}
    \end{bmatrix}
    \times
    \begin{bmatrix}
        b_{11} \\ b_{21} \\ b_{n1}
    \end{bmatrix}
    \; = \; a_{11}b_{11} + a_{12}b_{21} + \cdots + a_{1n}b_{n1}
\end{equation*}



\end{document}