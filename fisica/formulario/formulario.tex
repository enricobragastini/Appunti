\documentclass[a4paper,14pt,landscape]{extarticle}
\usepackage[
    margin=0.5cm
]{geometry}
\usepackage[utf8]{inputenc}
\usepackage[T1]{fontenc}
\usepackage[italian]{babel}
\usepackage{multicol}
\usepackage{amsmath}
\usepackage{titlesec}

\title{}
\author{}
\date{}

\pagestyle{empty}
\titlespacing*{\section}{0pt}{0pt}{0pt}
\titleformat{\section}{\large\bfseries}{\thesection}{1em}{}

\begin{document}
\begin{center}
    \vspace{1cm}
    \textbf{\Large Formulario Fisica 1}
    \vspace{1cm}
\end{center}

\begin{multicols*}{2}
    \section*{Moto Rettilineo Uniforme}
    \begin{itemize}
        \item $v_x = v_m = \text{costante}$
        \item $v_x = \frac{\Delta x}{\Delta t}$
        \item $x_f = x_i + v_x t$
    \end{itemize}

    \section*{Accelerazione media e istantanea}
    \begin{itemize}
        \item $a_m = \frac{\Delta v_x}{\Delta t} = \frac{v_{xf} - v_{xi}}{t_{f} - f_{i}}$
        \item $a_x = \displaystyle\lim_{\Delta t \to 0} \frac{\Delta v_x}{\Delta t} = \frac{dv_x}{dt} = \frac{d^2 x}{dt^2}$
    \end{itemize}

    \section*{Moto Rettilineo Uniformemente Accelerato}
    \begin{itemize}
        \begin{multicols*}{2}
            \item $a_m = a_x = \text{costanti}$
            \item $v_{xf} = v_{xi} + a_x t$
            \item $v_m = \tfrac{1}{2} (v_{xi} + v_{xf})$
            \item $x_f = x_i + v_{xi}t + \tfrac{1}{2} a_x t^2$
            \item $v^2_{xf} = v^2_{xi} + 2a_x(x_f - x_i)$
        \end{multicols*}
    \end{itemize}

    \section*{Corpi in caduta libera}
    \begin{itemize}
        \item $a_y = g = \text{costante}$
        \item $g = 9.81 m/s^2$
    \end{itemize}

    \section*{Moto dei Proiettili}
    \begin{itemize}
        \item $h_{max} = \frac{v_i^2 \sin^2 \theta_i}{2g}$
        \item $R \text{(gittata)} = \frac{v_i^2 \sin 2\theta_i}{g}$
    \end{itemize}

    \section*{Moto Circolare Uniforme}
    \begin{itemize}
        \begin{multicols*}{2}
            \item $a_c = \frac{v^2}{r}$
            \item $T = \frac{2 \pi r}{v}$
            \item $\omega = \frac{2\pi}{T}$
            \item $\omega = \frac{v}{r}$
            \item $v = r\omega$
            \item $a_c = r\omega^2$
        \end{multicols*}
    \end{itemize}

    \section*{Seconda legge di Newton}
    \begin{itemize}
        \item $\vec{a} \propto \frac{\sum \vec{F}}{m}$
        \item $\sum \vec{F} = m\vec{a}$
        \item $F_g (\text{forza peso}) = mg$
    \end{itemize}

    \section*{Attrito}
    \begin{itemize}
        \item Attrito statico: $f_s \leq \mu_s n$ \newline
              $\mu_s$ : coefficiente di attrito statico \newline
              $n$ : reazione vincolare
        \item Attrito dinamico: $f_k = \mu_k n$ \newline
              $\mu_k$ : coefficiente di attrito dinamico
    \end{itemize}

    \section*{Lavoro}
    \begin{itemize}
        \item $W \equiv F \Delta r \cos{\theta} = \int_{x_i}^{x_f} F_x \; dx = \vec{F} \cdot \Delta \vec{r}$
    \end{itemize}

    \section*{Prodotto scalare}
    \begin{itemize}
        \item $\vec{A} \cdot \vec{B} = AB\cos{\theta} = A_xB_x + A_yB_y + A_zB_z$
    \end{itemize}

    \section*{Lavoro di una molla}
    \begin{itemize}
        \item Legge di Hooke: $F_s = -kx$
        \item $W_s = \int_{x_i}^{x_f} (-kx) \; dx = \tfrac{1}{2}kx_i^2 - \tfrac{1}{2}kx_f^2$
    \end{itemize}

    \section*{Energia cinetica}
    \begin{itemize}
        \begin{multicols*}{3}
            \item $K \equiv \tfrac{1}{2} mv^2$
            \item $W = \Delta K$
            \item $K_f = K_i + W_{est}$
        \end{multicols*}
        \item $v = \sqrt{\tfrac{2K}{m}}$
    \end{itemize}

    \section*{Energia potenziale}
    \begin{itemize}
        \item Gravitazionale: $U_g \equiv mgy$
        \item $W_{est} = \Delta U_g = mgy_f - mgy_i$
        \item Elastica: $U_s \equiv \tfrac{1}{2} kx^2$
        \item $W_{F_{app}} = \Delta U_s = \tfrac{1}{2}kx_f^2 - \tfrac{1}{2}kx_i^2$
    \end{itemize}

    \section*{Conservazione dell'energia}
    \begin{itemize}
        \item $\Delta E_{sistema} = \sum T$
        \item $K_f + U_f = K_i + U_i$ (attenzione all'attrito!)
    \end{itemize}

    \section*{Sistemi con attrito dinamico}
    \begin{itemize}
        \begin{multicols*}{2}
            \item $\Delta K = -f_kd$
            \item $U_f + K_f - U_i - K_i = -f_kd$
        \end{multicols*}
    \end{itemize}

    \section*{Energia meccanica}
    \begin{itemize}
        \begin{multicols*}{2}
            \item $E_{mecc} = K + U$
            \item $E_{mecc_f} = E_{mecc_i}$
        \end{multicols*}
    \end{itemize}

    \section*{Potenza}
    \begin{itemize}
        \begin{multicols*}{3}
            \item $P \equiv \tfrac{dE}{dt}$
            \item $P_{media} \equiv \tfrac{W}{\Delta t}$
            \item $P_{ist} = \tfrac{dW}{dt} = \vec{F} \cdot \vec{v}$
        \end{multicols*}
    \end{itemize}

    \section*{Quantità di moto}
    \begin{itemize}
        \begin{multicols*}{2}
            \item $p \equiv m\vec{v}$
            \item $\sum \vec{F} = m\vec{a} = m \tfrac{d\vec{v}}{dt} = \tfrac{d\vec{p}}{dt}$
            \item $\Delta \vec{p} = \int_{t_i}^{t_f} \sum\vec{F} \; dt$
            \item Impulso: $\Delta \vec{p} = \vec{I} \;\; [N\cdot s]$          
        \end{multicols*}
    \end{itemize}

    \section*{Urti perfettamente anelastici}
    \begin{itemize}
        \item $\Delta \vec{p} = 0 \;\; \rightarrow \;\; m_1\vec{v}_1 + m_2\vec{v}_2 = (m_1+m_2)\vec{v}_f$
        \item $\vec{v}_f = \tfrac{m_1\vec{v}_{1i} + m_2\vec{v}_{2i}}{m_1 + m_2}$
    \end{itemize}

    \section*{Urti elastici}
    \begin{itemize}
        \item $\vec{p}_i = \vec{p}_f \;\; \text{e} \;\; K_i = K_f$
        \item $v_{1f} = \left( \tfrac{m_1 - m_2}{m_1 + m_2} \right) v_{1i} + \left( \tfrac{2m_2}{m_1 + m_2} \right) v_{2i}$
        \item $v_{2f} = \left( \tfrac{2m_1}{m_1 + m_2} \right) v_{1i} + \left( \tfrac{m_2 - m_1}{m_1 + m_2} \right) v_{2i}$
    \end{itemize}

    \section*{Centro di massa}
    \begin{itemize}
        \item $x_{CM} \equiv \tfrac{m_1x_1 + \cdots + m_nx_n}{m_1 + \cdots + m_n}$
        \begin{multicols*}{2}            
            \item $\vec{v}_{CM} = \tfrac{1}{M} \sum_i m_i\vec{v}_i$
            \item $\vec{a}_{CM} = \tfrac{1}{M} \sum_i m_i\vec{}_i$
        \end{multicols*}
        \item $x_{CM} = \tfrac{1}{M}\sum_i m_ix_i$ (analogo per y e z)
        \item $\vec{r}_{CM} = x_{CM}\hat{i} + y_{CM}\hat{j} + z_{CM}\hat{k} = \tfrac{1}{M} \sum_i m_i \vec{r}_i$
        \item Corpo esteso: $x_{CM} = \tfrac{1}{M} \int x \; dm$ (analogo per y e z)
        \item Corpo esteso: $\vec{r}_{CM} = \frac{1}{M} \int \vec{r} \; dm$
    \end{itemize}

    \section*{Rotazione Corpo Rigido}
    \begin{itemize}

        \begin{multicols*}{2}
            \item Posizione (angolare): $\theta$
            \item $\omega_{media} = \tfrac{\Delta \theta}{\Delta t}$
            \item $\omega_{ist} = \tfrac{d \theta}{d t}$
            \item Spost: $\Delta \theta = \theta_f - \theta_i$
            \item $\alpha_{media} = \tfrac{\Delta \omega}{\Delta t}$
            \item $\alpha_{ist} = \tfrac{d \omega}{d t}$
        \end{multicols*}
    \end{itemize}

    \section*{Corpo rigido con acc. ang. costante}
    \begin{itemize}
        \begin{multicols*}{2}
            \item $\omega_f = \omega_i + \alpha t$
            \item $\theta_f = \theta_i + \omega_i t + \tfrac{1}{2} \alpha t^2$ 
            \item $\omega_f^2 = \omega_i^2 + 2\alpha (\theta_f - \theta_i)$
            \item $\theta_f = \theta_i + \tfrac{1}{2}(\omega_i + \omega_f)t$
        \end{multicols*}
    \end{itemize}

    \section*{Variabili angolari e lineari}
    \begin{itemize}
        \begin{multicols*}{2}
            \item $v_t = r\omega$
            \item $a_t = r\alpha$
            \item $a_c = \tfrac{v^2}{r} = r\omega^2$
            \item $a = r\sqrt{\alpha^2 + \omega^4}$
        \end{multicols*}
    \end{itemize}

    \section*{Momento}
    \begin{itemize}
        \item $\tau = rF\sin{\phi} = Fd = r \times F$
        \item $d = r\sin{\phi}$
    \end{itemize}

    \section*{Prodotto vettoriale}
    \begin{itemize}
        \begin{multicols}{2}
        \item $C = A \times B \equiv AB\sin{\theta}$
        \end{multicols}
    \end{itemize}

    \section*{Momento risultante}
    \begin{itemize}
        \item $\sum \tau = (ma_t)r = (mr^2)\alpha = I\alpha$
    \end{itemize}

    \section*{Momento d'Inerzia}
    \begin{itemize}
        \item $I = \sum_i m_ir_i^2 = \int \rho r^2 dV$
        \item Teorema Assi Paralleli: $I = I_{CM} + MD^2$
    \end{itemize}

    \section*{Energie di rotazione}
    \begin{itemize}
        \item Cinetica: $K_R = \tfrac{1}{2} I\omega^2$
        \item Lavoro: $dW = (F\sin{\phi})rd\theta = \tau d\theta = \tfrac{1}{2} I\omega_f^2 - \tfrac{1}{2}I\omega_i^2$
        \item Potenza: $\tfrac{dW}{dt} = \tau\omega$
    \end{itemize}

    \section*{Momento angolare}
    \begin{itemize}
        \begin{multicols}{2}
            \item $\vec{L} = \vec{r} \times \vec{p}$
            \item $L_z = I\omega$         
        \end{multicols}
        \item $I_i\omega_i = I_f\omega_f = \text{costante}$   
    \end{itemize}

    \section*{Equilibrio corpo rigido}
    \begin{itemize}
        \item $\sum F_x = \sum F_y = \sum \tau_z = 0$
    \end{itemize}

    \section*{Gravitazione}
    \begin{itemize}
        \begin{multicols*}{2}
            \item $F_g = G \frac{m_1m_2}{r^2}$
            \item $G = 6.674 \times 10^{-11}$
            \item $g = \frac{GM_T}{r^2} = \frac{GM_T}{(R_T + h)^2}$
            \item Campo grav.: $\vec{g} \equiv \frac{\vec{F}_g}{m_0}$
            \item Raggio orbita: $r = \sqrt[3]{\frac{T^2\cdot G \cdot M}{4 \pi^2}}$
            \item Vel. fuga: $v = \sqrt{\tfrac{2GM_{T}}{R_{T}}}$
            \item $K = \tfrac{GMm}{2r}$
            \item $U = -\tfrac{GMm}{r}$
        \end{multicols*}
        \item Velocità orbitale: $\sqrt{\tfrac{GM}{r}} = \sqrt{\tfrac{GM}{R+h}}$
    \end{itemize}

\end{multicols*}

\end{document}