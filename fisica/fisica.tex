\documentclass[12pt,oneside]{book}

% PACCHETTI
\usepackage{tabto}              % strumento per inserire tab nel testo
\usepackage[                    % geometria della pagina
    a4paper,
    inner=2cm,
    outer=3cm,
    top=3cm,
    bottom=3cm,
    bindingoffset=1.2cm
]{geometry}
\usepackage[utf8]{inputenc}     % 3 pacchetti per l'italiano
\usepackage[italian]{babel}
\usepackage[T1]{fontenc}
\usepackage{titlesec}           % custom chapter titles

\usepackage{fancyhdr}


% INFORMAZIONI SUL DOCUMENTO
\title{\Large{\textbf{Fisica 1}}}
\author{Enrico Bragastini}
\titleformat{\chapter}[display]{\normalfont\bfseries}{}{0pt}{\LARGE}


% CONTENUTO
\begin{document}
\pagestyle{fancy}
\fancyhf{}
\lhead{\nouppercase\leftmark}
\cfoot{\thepage}
\frontmatter

% Prima pagina - Titolo
\maketitle
\tableofcontents

\mainmatter
\chapter{Nozioni di base}
\section{Misura di una grandezza}
La misura di una grandezza può avvenire mediante un dispositivo misurabile
oppure in confronto con un'altra grandezza fisica omogenea di riferimento costante.

L'espressione di una grandezza fisica avviene nella forma: Numero + \underline{Unità di Misura}

\section{Grandezze fisiche fondamentali e derivate}
\subsection{Grandezze fisiche fondamentali}
Le grandezze fisiche fondamentali sono:
\begin{itemize}
    \item Lunghezza                 \tabto{8cm} [L]
    \item Massa                     \tabto{8cm}  [M]
    \item Tempo                     \tabto{8cm}  [t]
    \item Intensità Di Corrente     \tabto{8cm}  [i]
    \item Temperatura Assoluta      \tabto{8cm}  [T]
\end{itemize}

\subsection{Grandezze fisiche derivate}
Le grandezze fisiche derivate sono:
\begin{itemize}
    \item Superficie
    \item Volume
    \item Velocità
    \item Accelerazione
    \item Forza
    \item Pressione
    \item ...
\end{itemize}

\section{Sistemi di Unità di Misura}
\begin{center}
    \bgroup
    \def\arraystretch{1.5}
    \begin{tabular}{ |c| c c c c c|}
        \hline
        SISTEMA     & Lunghezza & Massa & Tempo & Corrente & Temperatura \\
        \hline
        MKS (s. i.) & m         & kg    & s     & A        & °K          \\
        \hline
        cgs         & cm        & g     & s     & A        & °K          \\
        \hline
    \end{tabular}
    \egroup
\end{center}

\subsection{Ulteriori Unità di Misura}
Esistono ulteriori sistemi di unità di misura che permettono di avere maggiore comodità
nelle misurazioni di particolari grandezze.
Se ne elencano alcuni:

\begin{enumerate}
    \item Lunghezza:    \tabto{3cm} Ångströms, Anno-Luce
    \item Tempo:        \tabto{3cm} Minuto, Ora
    \item Volume:       \tabto{3cm} Litro
    \item Velocità:     \tabto{3cm} Chilometro/Ora
    \item Pressione:    \tabto{3cm} Atmosfera, Millimetro di mercurio
    \item Energia:      \tabto{3cm} Elettrovolt, Chilovattora
\end{enumerate}

\section{Notazione Scientifica}
Per i numeri particolarmente grandi o piccoli risulta comodo rappresentarli
in \textbf{Notazione Scientifica} utilizzando le potenze del 10.

(Da integrare)

\section{Analisi Dimensionale}
L'analisi dimensionale è utile per controllare che una formula sia giusta o per dedurre come deve
essere una certa formula.
Le dimensioni possono essere trattate come quantità algebriche. Due membri della stessa equazione devono avere le
stesse dimensioni.

(Da integrare)

\section{Coordinate Cartesiane}
\section{Coordinate polari}

\section{Grandezze fisiche scalari e vettoriali}


\end{document}